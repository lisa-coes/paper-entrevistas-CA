% Options for packages loaded elsewhere
\PassOptionsToPackage{unicode}{hyperref}
\PassOptionsToPackage{hyphens}{url}
\PassOptionsToPackage{dvipsnames,svgnames,x11names}{xcolor}
%
\documentclass[
  letterpaper,
  DIV=11,
  numbers=noendperiod]{scrreprt}

\usepackage{amsmath,amssymb}
\usepackage{lmodern}
\usepackage{iftex}
\ifPDFTeX
  \usepackage[T1]{fontenc}
  \usepackage[utf8]{inputenc}
  \usepackage{textcomp} % provide euro and other symbols
\else % if luatex or xetex
  \usepackage{unicode-math}
  \defaultfontfeatures{Scale=MatchLowercase}
  \defaultfontfeatures[\rmfamily]{Ligatures=TeX,Scale=1}
\fi
% Use upquote if available, for straight quotes in verbatim environments
\IfFileExists{upquote.sty}{\usepackage{upquote}}{}
\IfFileExists{microtype.sty}{% use microtype if available
  \usepackage[]{microtype}
  \UseMicrotypeSet[protrusion]{basicmath} % disable protrusion for tt fonts
}{}
\makeatletter
\@ifundefined{KOMAClassName}{% if non-KOMA class
  \IfFileExists{parskip.sty}{%
    \usepackage{parskip}
  }{% else
    \setlength{\parindent}{0pt}
    \setlength{\parskip}{6pt plus 2pt minus 1pt}}
}{% if KOMA class
  \KOMAoptions{parskip=half}}
\makeatother
\usepackage{xcolor}
\setlength{\emergencystretch}{3em} % prevent overfull lines
\setcounter{secnumdepth}{5}
% Make \paragraph and \subparagraph free-standing
\ifx\paragraph\undefined\else
  \let\oldparagraph\paragraph
  \renewcommand{\paragraph}[1]{\oldparagraph{#1}\mbox{}}
\fi
\ifx\subparagraph\undefined\else
  \let\oldsubparagraph\subparagraph
  \renewcommand{\subparagraph}[1]{\oldsubparagraph{#1}\mbox{}}
\fi


\providecommand{\tightlist}{%
  \setlength{\itemsep}{0pt}\setlength{\parskip}{0pt}}\usepackage{longtable,booktabs,array}
\usepackage{calc} % for calculating minipage widths
% Correct order of tables after \paragraph or \subparagraph
\usepackage{etoolbox}
\makeatletter
\patchcmd\longtable{\par}{\if@noskipsec\mbox{}\fi\par}{}{}
\makeatother
% Allow footnotes in longtable head/foot
\IfFileExists{footnotehyper.sty}{\usepackage{footnotehyper}}{\usepackage{footnote}}
\makesavenoteenv{longtable}
\usepackage{graphicx}
\makeatletter
\def\maxwidth{\ifdim\Gin@nat@width>\linewidth\linewidth\else\Gin@nat@width\fi}
\def\maxheight{\ifdim\Gin@nat@height>\textheight\textheight\else\Gin@nat@height\fi}
\makeatother
% Scale images if necessary, so that they will not overflow the page
% margins by default, and it is still possible to overwrite the defaults
% using explicit options in \includegraphics[width, height, ...]{}
\setkeys{Gin}{width=\maxwidth,height=\maxheight,keepaspectratio}
% Set default figure placement to htbp
\makeatletter
\def\fps@figure{htbp}
\makeatother
\newlength{\cslhangindent}
\setlength{\cslhangindent}{1.5em}
\newlength{\csllabelwidth}
\setlength{\csllabelwidth}{3em}
\newlength{\cslentryspacingunit} % times entry-spacing
\setlength{\cslentryspacingunit}{\parskip}
\newenvironment{CSLReferences}[2] % #1 hanging-ident, #2 entry spacing
 {% don't indent paragraphs
  \setlength{\parindent}{0pt}
  % turn on hanging indent if param 1 is 1
  \ifodd #1
  \let\oldpar\par
  \def\par{\hangindent=\cslhangindent\oldpar}
  \fi
  % set entry spacing
  \setlength{\parskip}{#2\cslentryspacingunit}
 }%
 {}
\usepackage{calc}
\newcommand{\CSLBlock}[1]{#1\hfill\break}
\newcommand{\CSLLeftMargin}[1]{\parbox[t]{\csllabelwidth}{#1}}
\newcommand{\CSLRightInline}[1]{\parbox[t]{\linewidth - \csllabelwidth}{#1}\break}
\newcommand{\CSLIndent}[1]{\hspace{\cslhangindent}#1}

\KOMAoption{captions}{tableheading}
\makeatletter
\makeatother
\makeatletter
\@ifpackageloaded{bookmark}{}{\usepackage{bookmark}}
\makeatother
\makeatletter
\@ifpackageloaded{caption}{}{\usepackage{caption}}
\AtBeginDocument{%
\ifdefined\contentsname
  \renewcommand*\contentsname{Table of contents}
\else
  \newcommand\contentsname{Table of contents}
\fi
\ifdefined\listfigurename
  \renewcommand*\listfigurename{List of Figures}
\else
  \newcommand\listfigurename{List of Figures}
\fi
\ifdefined\listtablename
  \renewcommand*\listtablename{List of Tables}
\else
  \newcommand\listtablename{List of Tables}
\fi
\ifdefined\figurename
  \renewcommand*\figurename{Figure}
\else
  \newcommand\figurename{Figure}
\fi
\ifdefined\tablename
  \renewcommand*\tablename{Table}
\else
  \newcommand\tablename{Table}
\fi
}
\@ifpackageloaded{float}{}{\usepackage{float}}
\floatstyle{ruled}
\@ifundefined{c@chapter}{\newfloat{codelisting}{h}{lop}}{\newfloat{codelisting}{h}{lop}[chapter]}
\floatname{codelisting}{Listing}
\newcommand*\listoflistings{\listof{codelisting}{List of Listings}}
\makeatother
\makeatletter
\@ifpackageloaded{caption}{}{\usepackage{caption}}
\@ifpackageloaded{subcaption}{}{\usepackage{subcaption}}
\makeatother
\makeatletter
\@ifpackageloaded{tcolorbox}{}{\usepackage[many]{tcolorbox}}
\makeatother
\makeatletter
\@ifundefined{shadecolor}{\definecolor{shadecolor}{rgb}{.97, .97, .97}}
\makeatother
\makeatletter
\makeatother
\ifLuaTeX
  \usepackage{selnolig}  % disable illegal ligatures
\fi
\IfFileExists{bookmark.sty}{\usepackage{bookmark}}{\usepackage{hyperref}}
\IfFileExists{xurl.sty}{\usepackage{xurl}}{} % add URL line breaks if available
\urlstyle{same} % disable monospaced font for URLs
\hypersetup{
  pdftitle={paper-entrevistas-CA},
  pdfauthor={LISA - COES},
  colorlinks=true,
  linkcolor={blue},
  filecolor={Maroon},
  citecolor={Blue},
  urlcolor={Blue},
  pdfcreator={LaTeX via pandoc}}

\title{paper-entrevistas-CA}
\author{LISA - COES}
\date{4/20/23}

\begin{document}
\maketitle
\ifdefined\Shaded\renewenvironment{Shaded}{\begin{tcolorbox}[breakable, borderline west={3pt}{0pt}{shadecolor}, interior hidden, boxrule=0pt, sharp corners, enhanced, frame hidden]}{\end{tcolorbox}}\fi

\renewcommand*\contentsname{Table of contents}
{
\hypersetup{linkcolor=}
\setcounter{tocdepth}{2}
\tableofcontents
}
\bookmarksetup{startatroot}

\hypertarget{preface}{%
\chapter*{Preface}\label{preface}}
\addcontentsline{toc}{chapter}{Preface}

\markboth{Preface}{Preface}

Este documento contiene el paper cualitativo sobre prácticas, creencias
y actitudes sobre Ciencia Abierta en académicos de las Ciencias Sociales
en Chile. El trabajo de campo de esta fase cualitativa fue desarrollado
entre julio y octubre del 2022. En total se entrevistaron a 14
académicos/as de Ciencias Sociales, con representación de distintas
disciplinas (Psicología, Sociología, Antropología, Economía) y métodos
(cuantitativos y cualitativos). Siempre resguardando representatividad
por género y que realizaran investigación empírica.

El repositorio está creado a partir de la plantilla de construcción de
Quarto books. Para más información, visitar
\url{https://quarto.org/docs/books}.

\bookmarksetup{startatroot}

\hypertarget{introducciuxf3n}{%
\chapter{Introducción}\label{introducciuxf3n}}

Durante los últimos años, el mundo académico ha experimentado una serie
de desafíos relacionados con el concepto de apertura, que pueden
describirse en dos aspectos principales. El primero se refiere a la
llamada crisis de replicación (Baker, 2016; Nosek et al., 2015; Peng,
2015), referida a las dificultades para replicar los resultados de la
investigación debido a la falta de transparencia en el proceso de
investigación, donde se ha sido posible evidenciar importantes
variaciones en los resultados en equipos de investigación con datos
idénticos (Breznau et al., 2021). Esto ha tenido como consecuencia no
sólo el fracaso en la replicación de muchos hallazgos publicados, sino
también el descubrimiento y la denuncia de datos y resultados falsos
(Chopik et al., 2018) con el fin de lograr publicar en revistas de alto
impacto. El segundo desafío proviene de la apertura en términos de
acceso, mediante el cual varias comunidades académicas han reaccionado
contra las altas barreras de pago y el modelo de negocio impuesto por
las compañías editoriales para acceder a los productos de la
investigación científica: las universidades pagan una suscripción para
acceder al trabajo incluso de sus propios investigadores, y a su vez el
público fuera de la academia debe pagar nuevamente por obtener productos
financiados por sus impuestos. Una de las reacciones más comentadas ha
sido la cancelación de las suscripciones a revistas como Elsevier por
parte de grandes universidades como UCLA, lo cual posteriormente se
resolvió a través de un acuerdo basado en la adopción y promoción de
prácticas en Open Access.

Las barreras en la transparencia y el acceso son síntomas de una cultura
académica presionada por la publicación de indicadores que están
sesgados hacia resultados ``significativos'' en términos estadísticos,
lo que lleva a una tendencia a forzar los resultados (p-hacking),
llegando incluso a manipular y falsear datos para confirmar las
hipótesis propuestas (Head et al., 2015) o también el establecer
hipótesis ad-hoc luego de conocer los resultados de un estudio
(Hollenbeck \& Wright, 2017; Kerr, 1998). Esto tiene como consecuencia
que finalmente el principal público objetivo de la ciencia son los
editores de revistas de alto rango, dejando de lado a otros públicos
como la sociedad civil, el estado y la ciudadanía. Tal escenario es
particularmente sensible para las ciencias sociales, que se supone basan
sus estudios en problemas sociales relevantes para las personas, las
comunidades y las sociedades en general.

Un número creciente de iniciativas en todo el mundo están abordando
temas de replicabilidad, transparencia y acceso en la ciencia, como el
Centro para la Ciencia Abierta (COS), la Iniciativa de Berkeley para la
Transparencia en Ciencias Sociales (BITSS) y el proyecto Teaching
Integrity in Empirical Research (TIER). Estas iniciativas fomentan la
apertura en diferentes etapas del proceso de investigación, como la
transparencia de los diseños de investigación a través de prerregistros
de estudios, la reproducibilidad de los análisis y la manipulación de
datos, así como en la publicación de preimpresiones libres de barreras
de pago. Muchas de estas prácticas han sido adoptadas por revistas como
recomendaciones o incluso requisitos para su publicación, así como
promovidas por instituciones científicas gubernamentales. Todo esto
implica un gran cambio en la forma de concebir, hacer y enseñar ciencia.
Además, parece un paso necesario para hacer que la ciencia sea más
relevante y cercana a quienes están fuera de la academia. En otras
palabras, sería difícil mejorar la apertura y el intercambio con las
comunidades locales si el trabajo dentro de la academia es mayormente
cerrado y no colaborativo.

Como muchos otros desarrollos en la ciencia, el movimiento de la ciencia
abierta ha llegado lentamente a América Latina, particularmente en las
ciencias sociales. Aunque ha habido algunas iniciativas en los últimos
años (como el Congreso de Ciencia Abierta y Ciudadanía en Argentina
2018, OpenCon LatAm Colombia 2019), la mayoría de ellas son promovidas
desde las ciencias naturales. En esta línea, a partir del año 2021 la
Agencia Nacional de Investigación y Desarrollo (ANID) realizará la
implementación de una \textbf{Política de acceso abierto de información
científica y datos de investigación}, la cual busca establecer un
estándar de acceso público a los productos de investigación que hayan
sido financiados por recursos de ANID, por lo tanto es de carácter
estratégico desarrollar iniciativas que se propongan contribuir a la
apertura de la ciencia.

A partir de este diagnóstico, el presente proyecto tiene por objetivo
analizar el conocimiento, creencias y prácticas de ciencia abierta en
académic\_s de ciencias sociales en Chile, y desde este análisis generar
recomendaciones y propuestas tanto para el quehacer académico como a las
políticas científicas.

\bookmarksetup{startatroot}

\hypertarget{antecedentes}{%
\chapter{Antecedentes}\label{antecedentes}}

\begin{enumerate}
\def\labelenumi{\alph{enumi})}
\item
  Revisión ciencia abierta en general (Kevin, práctica)
\item
  Principales componentes de la ciencia abierta (JC - modelo LISA)
\item
  Antecedentes políticas de ciencia abierta Chile
\end{enumerate}

\bookmarksetup{startatroot}

\hypertarget{muxe9todo}{%
\chapter{Método}\label{muxe9todo}}

Se realizaron 13 entrevistas semiestructuradas a investigadores
empíricos de ciencias sociales, seleccionados a partir de un muestreo
aleatorio simple por criterios. El marco muestral consideró a todos los
académicos que se hubiesen adjudicado un Proyecto Fondecyt Regular entre
2018 y 2019, en los grupos de estudio Antropología y Arqueología,
Ciencias económicas y administrativas, Ciencias jurídicas y políticas,
Sicología y Sociología. Considerando cuotas por sexo y grupo de estudio,
se seleccionaron aleatoriamente uno o dos informantes por sexo para cada
grupo de estudio. Se excluyeron los casos en que los investigadores
seleccionados se abocaran estrictamente a la investigación teórica.
Asimismo, se excluyeron casos como: a) investigadores del grupo de
estudio Ciencias jurídicas y políticas dedicados a la investigación
filosófica; b) investigadores del grupo de estudio Antropología y
Arqueología no dedicados a la antropología social; c) investigadores del
grupo de estudio Sicología abocados a la neurociencia. También, dado que
el grupo de estudio de sociología está compuesto por investigadores de
otras disciplinas, como la psicología social y el trabajo social, se
aleatorizó hasta asegurar la selección de, al menos, un sociólogo. La
siguiente tabla da cuenta de la composición final de la muestra:

\begin{longtable}[]{@{}lll@{}}
\toprule()
Grupo de estudio & Hombre & Mujer \\
\midrule()
\endhead
Antropología y arqueología (Antropología) & 0 & 2 \\
Ciencias económicas y administrativas (Economía) & 1 & 1 \\
Ciencias jurídicas y políticas (Ciencia política) & 2 & 1 \\
Psicología & 1 & 2 \\
Sociología & 1 & 2 \\
\bottomrule()
\end{longtable}

Las entrevistas semiestructuradas se realizaron a través de
videollamada, con una duración entre cuarenta minutos y una hora. Se
abordaron tópicos relativos a la familiarización y conocimiento de los
investigadores respecto a la ciencia abierta, su participación en
prácticas de ciencia abierta, sus actitudes y valoraciones frente a
prácticas de apertura y transparencia, así como frente a una futura
posible obligatoriedad de tales, entre otros.

Se empleó análisis temático, en tanto es un método que permite
identificar y describir patrones de significados (Braun \& Clarke 2006),
así como el conjunto de relaciones y jerarquías que entre estos se
generan, organizados en torno al concepto de ciencia abierta y las ideas
asociadas a este, a partir de códigos tanto semánticos como latentes
(Boyatzis, 1998). En este sentido, este método es particularmente útil
para comprender los puntos de vista, conocimientos, experiencias y
valores de los participantes respecto a la ciencia abierta, a partir del
conjunto de datos.

El análisis fue realizado principalmente por dos investigadores, apoyado
por el software de análisis cualitativo ATLAS.ti en su versión 9.0.5. El
proceso de codificación fue realizado simultáneamente por ambos
investigadores y discutido en reuniones periódicas con el equipo de
investigación, con fines de asegurar triangulación entre investigadores
a lo largo del proceso de análisis (Nowell et al., 2017). Asimismo, se
archivaron sistemáticamente versiones del proyecto y sus productos, a la
vez que se llevó el registro de las reflexiones personales de los
investigadores y las discusiones de equipo en torno al análisis en un
diario reflexivo, para poder resguardar la auditabilidad del proceso de
investigación (Silver y Lewins, 2007).

\bookmarksetup{startatroot}

\hypertarget{anuxe1lisis}{%
\chapter{Análisis}\label{anuxe1lisis}}

\textbf{Un concepto difuso}

Inicialmente el concepto de ciencia abierta aparece como una idea
difícil de definir, aparentemente lejana para los participantes.
Enfrentados a la primera pregunta de la entrevista, `¿Está familiarizado
con el tema de la ciencia abierta?', la mayoría de los participantes
proporcionaron una respuesta insegura o redirigieron la pregunta al
investigador, varios reconociendo que escasamente habían escuchado
hablar del tema. Esto, sin embargo, coexiste con diferentes grados de
conocimiento de los problemas que enfrenta la ciencia abierta, prácticas
comunes de apertura y transparencia y, en ocasiones, experiencias de
investigadores que habían participado directamente en una serie de
iniciativas institucionales o internacionales de ciencia abierta. Por
ejemplo, se han propuesto protocolos de pre-registro y apertura de datos
para ciertos tipos de investigaciones (Ciencia política, métodos
mixtos). En este sentido, señalan que están poco familiarizados con la
idea de ciencia abierta, y no asocian necesariamente el concepto de
ciencia abierta a prácticas e iniciativas de ciencia abierta concretas.

En lugar de mencionar prácticas concretas de ciencia abierta, el
concepto es asociado principalmente a una variedad de iniciativas y
orientaciones normativas de la investigación. Se asocia fuertemente el
concepto con iniciativas de vinculación con el medio y divulgación
científica, así como con el acceso de la población general a los
resultados de investigación, particularmente ``las comunidades'' y el
público extra-académico. También algunos entrevistados asocian el
concepto de ciencia abierta a la producción de conocimiento, refiriendo
al involucramiento de participantes, comunidades y grupos de interés en
el proceso de investigación; el empleo de métodos mixtos (Sociología,
cualitativos) o la interdisciplinariedad (Antropología, cualitativos;
Sociología, cualitativos). Sólo unos pocos hacen referencia a prácticas
como la apertura de datos o la publicación en revistas open-access.

Si bien el concepto es difícil de delimitar en los discursos de los
participantes, queda en evidencia una valoración general positiva de la
idea de la ciencia abierta. Exceptuando a un par de investigadores que
hacen referencia a la ciencia abierta como una ``fiebre'' extendida o
como un ``proyecto demasiado ambicioso'', las valoraciones iniciales
suelen ser a favor de la ciencia abierta. Algunas de estas valoraciones
son vagamente positivas, mientras que en otros casos se relaciona
directamente la idea de ciencia abierta con valores democráticos,
señalando que la ciencia abierta avanza hacia la democratización del
conocimiento y la ciencia. Cabe destacar que la mayor parte de las
valoraciones positivas respecto a la ciencia abierta son expresadas por
los participantes al comienzo de la entrevista, pero luego se presentan
opiniones ambivalentes o dubitativas, e incluso contradiciéndose al
concluir la entrevista. Esto sobre todo a raíz de diversas
preocupaciones asociadas a una posible obligatoriedad de las prácticas
de ciencia abierta, lo cual se abordará posteriormente.

\textbf{Ambivalencia sobre la crisis de apertura}

Si bien no se menciona explícitamente la existencia de una crisis de
apertura del acceso a las publicaciones científicas, a la vez que en
pocas ocasiones se mencionan las malas prácticas académicas como una
problemática extendida, existe un acuerdo transversal respecto al
diagnóstico de un poder editorial mundial con un rol predominante de los
capitales del norte global, que coarta el libre acceso a los productos
de investigación. Desde esta base, hay quienes perciben que la situación
del poder editorial tiene efectos importantes y, por lo tanto,
diagnostican una crisis de acceso, mientras otros perciben que los
efectos son mínimos o no presentan un problema práctico para la
producción científica.

Entre quienes consideran que la crisis de acceso es problemática, se
señala que esta genera desigualdades entre investigadores, tanto entre
el Norte y el Sur global, como entre investigadores pertenecientes a
instituciones con menores recursos, como podría ser el caso de los
investigadores de centros de estudios pequeños o de universidades
regionales.

\begin{verbatim}
(...) los precios son absurdos, muy absurdos. Son para alguien que vive en Chicago o en Suiza, y que gana como tal (...) Es privativo, para nosotros es absurdo. (Psicología, cuantitativa)

Y nuestras universidades del sur, sur global, no tienen acceso a estas revistas, porque estas revistas son en dólares o en euros y el pack de acceso a esas revistas, que está vendido por estas empresas, es carísimo. Nuestras universidades, la gran parte, no pueden pagar, especialmente las universidades regionales (...) (Antropología, cualitativa)
\end{verbatim}

Por otro lado, se reconoce que la crisis de apertura tiene consecuencias
negativas para el presupuesto público y el desarrollo desde la ciencia,
especialmente en los países del Sur. Se cuestiona el sistema editorial
como uno que no es sólo injusto para los investigadores en el campo,
sino también para las naciones. Esta situación es catalogada como
absurda, vergonzosa e inexplicable, como ilustran las siguientes citas:

\begin{verbatim}
(...) hay situaciones que sí me parecen claramente absurdas (...) que después ciertas organizaciones locales quieran acceder a ese conocimiento financiado por el Estado de Chile, y se les cobre por algo que ellos mismos han pagado desde sus impuestos para producir conocimiento para el desarrollo nacional. (Sociología, cualitativa)

lo que estamos viviendo ahora es un mecanismo de transferencia de capital de los estados del sur global al norte global vergonzoso (...) Usando plata de los contribuyentes chilenos para financiar la publicación de artículos científicos por empresas que después venden estos artículos a las universidades chilenas, al precio nada barato de 65 dólares el texto (Antropología, cualitativa)
\end{verbatim}

Sin embargo, la existencia de este sistema de transferencia global no es
visto por todos los investigadores como una situación crítica. En
contraste, algunos investigadores señalaron que las instituciones están
avanzando para mejorar el acceso a través de los sistemas de
bibliotecas, percibiendo que sus propias instituciones les permiten
acceder a toda la información que necesitan.

Otros señalan que en América Latina las prácticas de acceder a productos
científicos como pre-print o working paper, a través de mecanismos
informales o mediante piratería, son extendidas entre los investigadores
y permiten hacer frente al problema del acceso sin mayores
consecuencias.

\begin{verbatim}
A ver, todas estas revistas... todos tenemos formas de ir por el costado, ¿no? (Ciencia política, métodos mixtos)

Bueno, al final, por un lado o por otro, uno consigue los artículos que le interesan. Ya sea por contacto directo con, o porque tiene un amigo que está en no-sé-dónde, y yo creo que en mi, ya más o menos larga experiencia en investigación, creo que una vez pagué por un artículo, hace como 20 años, una época en que internet no estaba tan desarrollado (Psicología, cuantitativo)
\end{verbatim}

\textbf{Actitudes, creencias y prácticas de Ciencia Abierta}

\begin{verbatim}
a) Apertura de resultados
\end{verbatim}

Lo primero a destacar es que hay consenso respecto a que la apertura de
los resultados debiese ser prioridad en lo que respecta al avance en
ciencia abierta. El compartir publicaciones de manera formal, por
ejemplo, publicando en revistas Open Access o con Ruta Dorada
(Psicología, cuantitativo; Antropología, cualitativo); o informal, por
medio de la publicación o la posibilidad de solicitud de versiones
finales o working paper o pre-print de las publicaciones en redes
sociales académicas como Academia.edu o ResearchGate, en Twitter o en
páginas web propias, son prácticas extendidas entre la gran mayoría de
los entrevistados. Asimismo, hay una creencia general respecto de que la
apertura de resultados es igualmente exigible a investigaciones
cuantitativas y cualitativas, sobre todo cuando estas son financiadas
con fondos públicos.

En relación con la iniciativa de compartir y solicitar artículos o
versiones previas a la publicación oficial en redes sociales académicas,
destacan dos actitudes por parte de los investigadores. Primero, aquella
según la cual esas iniciativas son cuestionables, en tanto los
investigadores se fundamentarían en razones entre el egoísmo y el
altruismo (Psicología, cuantitativos; Sociología, cualitativos) a la
hora de establecer tales relaciones de intercambio en estas redes. Ello
se debe a que esta actitud moralmente deseable de colaboración
científica permite, a la vez, conseguir un mayor número de citas, lo
cual va en la línea de los requisitos de financiamiento de investigación
que se abordarán posteriormente. Se percibe, entonces, la idea de que el
altruismo de compartir las publicaciones tiende a verse colonizado por
la racionalidad de los requisitos que impone la política de
financiamiento a la producción científica. Esto entrega una imagen
parcialmente negativa frente a la práctica, por lo que algunos
investigadores fundamentan el no compartir sus productos académicos en
estas plataformas, por demostrar que no comparten estas motivaciones
egoístas

\begin{verbatim}
(...) a uno le conviene, entre comillas, egoístamente, compartir tus datos y compartir tus códigos. Porque, voy a decirlo brutalmente, te van a citar más. Tú vas a ser más, tu trabajo va a ser más divulgado y mejor divulgado. (Psicología, cuantitativos)

No tengo, no tengo la pretensión de hacerme famoso ni que me citen mucho. Para mí mi trabajo se termina publicando. Después... qué sé yo. (Ciencia política, métodos mixtos)
\end{verbatim}

Por otro lado, hay casos en los que esta práctica se realiza con ciertas
preocupaciones, en la medida que se conoce la existencia de la cláusula
de no difusión que suelen integrar las revistas con barreras de pago al
acceso. De ese modo, una sensación de que la fiscalización no es lo
suficientemente fuerte como para ser problemática (Economía,
cuantitativos) convive con la preocupación de ser amenazados por las
revistas en caso de que se den cuenta de que realizan estas prácticas.
Tal temor se ve reforzado por las conversaciones que los investigadores
mantienen con sus pares respecto del respeto de la cláusula de no
difusión (Psicología, cuantitativos), sin que ningún investigador
entrevistado declarase haber sido fiscalizado efectivamente por haber
publicado sus investigadores en estos medios. Sin embargo, la alta
valoración positiva de la apertura de resultados motiva a los
investigadores a compartir sus publicaciones en estas redes sociales
pese a los posibles problemas legales que ello pueda producir.

Así, siguiendo con el ánimo de hacer accesibles los resultados de sus
proyectos de investigación, los investigadores han buscado medios
legales y formales para lograrlo. Uno de estos ha sido la Ruta Dorada,
alternativa que presentan algunas revistas para, contra pago del
investigador, permitir que una publicación sea de libre acceso.

En general, la experiencia de publicar pagando la Ruta Dorada es
descrita negativamente. Algunos investigadores han destacado que no
volverían a pagar para que una de sus publicaciones sea de libre acceso
(Psicología, cuantitativos; Antropología, cualitativos). Esto se debe,
principalmente, debido al alza en las tarifas de apertura, que se han
tendido a multiplicar en los últimos años, lo cual se ha calificado como
absurdo (Psicología, cuantitativos). Esto ha motivado a los
investigadores a cuestionar a los investigadores qué tan verdaderamente
se encuentran estas iniciativas enmarcadas en los principios de la
ciencia abierta, entre los cuales destaca el enfrentamiento al gran
poder de mercado de las compañías editoriales. En ese sentido, se
destaca el hecho de que los costos han pasado de los usuarios a los
investigadores (Psicología, cuantitativos). No obstante, algunos
investigadores señalan que muchas revistas basan su funcionamiento en el
referato gratuito, de modo que alguien debe hacerse cargo de los costos
que implica la apertura de las publicaciones en la medida que la
principal fuente de ingresos de muchas compañías editoriales está en la
venta del acceso a las publicaciones (Ciencia política, métodos mixtos).

Otra vía a través de la cual los investigadores buscan permitir que los
resultados que han producido tengan impacto social es la publicación de
columnas y editoriales en medios de prensa escrita (Psicología,
cuantitativo), creación de documentales (Sociología, cualitativo),
instancias de devolución con las comunidades investigadas u otras con
características similares (Antropología, cualitativo; Psicología,
cuantitativo), entre otros. Se destacan las tensiones que pueden existir
entre el ranking de impacto de las revistas en que se publica, y el
impacto real que tienen las investigaciones en las sociedades que
analizan. Destaca el relato de una investigadora que, luego de publicar
una investigación en una revista abierta en español pero de bajo
impacto, fue invitada por el Gobierno para asesorar políticas de su área
investigativa, lo cual no hubiese pasado de haber publicado tales
resultados en una revista con barreras de pago (Psicología,
cuantitativo).

En esa línea, es posible identificar actitudes generalmente negativas
hacia la política de difusión de resultados de ANID, con base en las
percepciones de deficiencia de esta a la hora de lograr volver público
el conocimiento producido con fondos públicos. Los entrevistados dan
cuenta de la rigidez de la política, en la medida que dificulta la
realización de ciertas iniciativas de difusión dirigidas a público
extra-académico, como la publicación de Podcasts. Además, se releva que,
si bien en ciertos grupos de estudio como Sociología y Ciencias de la
Información productos de difusión como documentales son bien
considerados, la estructura de financiamiento sigue otorgando un rol
central a la publicación en revistas indexadas a Web of Science (WOS) y
Scopus. Lo mismo sucede con publicar en revistas de libre acceso en
español que, en general, tienden a tener un bajo impacto académico pese
a ser más accesibles para el público en general.

Ello tiene como consecuencia que los resultados de mayor relevancia
científica se destinen a revistas con barreras de pago, a las cuales no
puede acceder gran parte del público. A su vez, ello conlleva una
``transferencia de capital del sur al norte global'' (Antropología,
cualitativos), en la medida que investigaciones financiadas con recursos
estatales son publicadas en revistas con barreras de pago, a las cuales
los investigadores nacionales o sus universidades deberán pagar - en
general, con fondos públicos - para poder acceder a ellos.

Los entrevistados constatan, en ese sentido, tensiones entre el
imperativo de avanzar en la apertura y difusión de los resultados
generados en las investigaciones financiadas con recursos públicos, la
rigidez de la política de difusión de ANID, y la estructura de
financiamiento e incentivos de esta última, en un contexto en que las
compañías editoriales presentan un ``alto poder de mercado'' (Economía,
cuantitativos). Así, para cumplir con la presión por publicar en
revistas de alto impacto producida por la estructura de financiamiento
de ANID, así como con los principios de apertura y difusión a los cuales
adscriben, los investigadores han desarrollado diversas técnicas para
diversificar los productos que publican. Una de ellas es la quema o
sacrificio de publicaciones, la cual consiste en diferir los resultados
generados en el proceso investigativo, enviando unos a revistas de mayor
impacto ---que otorgan un mayor puntaje a la hora de postular a fondos
de investigación---, así como otros a revistas locales, abiertas y,
muchas veces, en español:

\begin{verbatim}
[...] publicamos en una revista que es... Creo que es sólo SciELO [...] He publicado cosas mucho mejores, digamos, en mejores revistas que esa, y nunca me habían llamado del Ministerio [...] Y por este artículo nos llamaron, nos pidieron una entrevista, justamente ayer tuvimos que hacer una presentación para todos los equipos que trabajaban en la temática en la que nosotros investigamos. Entonces [...] es necesario hacer ese esfuerzo de, de repente, sacrificar, entre comillas, algunos datos y publicarlos en revistas que tienen menor indexación, porque finalmente esas son las que se difunden más y llegan más a donde uno... o al menos donde yo esperaría que llegaran, que es donde se pueden tomar decisiones [...] (Psicología, cuantitativos)
\end{verbatim}

En lo que respecta al acceso a publicaciones científicas, los
entrevistados han destacado el papel que desempeñan poseen las
universidades eny que se desempeñan a través de sus bibliotecas. Por
ello, y como se señaló en apartados anteriores, los entrevistados
tienden a plantear que no han tenido dificultades para acceder a
publicaciones. Es posible, sin embargo, constatar una diferencia:
mientras que investigadores vinculados a universidades prestigiosas y de
la Región Metropolitana señalan que nunca han tenido dificultades de
acceso en tanto la biblioteca de su universidad les provee toda la
literatura a la cual necesitan acceder; los investigadores que trabajan
en universidades ubicadas en otras regiones del país destacan, más bien,
la ampliación que han experimentado de manera progresiva gracias a la
inversión y al trabajo que se ha realizado por parte de las bibliotecas
de sus instituciones educativas.

\begin{enumerate}
\def\labelenumi{\alph{enumi})}
\setcounter{enumi}{1}
\tightlist
\item
  Transparencia del diseño
\end{enumerate}

Sólo algunos entrevistados declararon haber publicado un pre-registro de
sus investigaciones en la web Open Science Framework (Psicología,
cuantitativos; Ciencia política, métodos mixtos), o bien, a partir de
iniciativas levantadas por revistas (Ciencia política, métodos mixtos;
Economía, cuantitativos). En el caso de economía se declara que el
pre-registro es una práctica extendida y centralizada en los protocolos
generados por la American Economic Association, lo cual ha sido evaluado
de forma positiva pero, en algunos casos, como un estándar excesivo.
Esto último, sobre todo en la medida en que las revistas de la
disciplina han tendido a estandarizar el pre-registro de hipótesis como
requisito para publicar resultados de investigación. Asimismo, destaca
el que un entrevistado haya enviado a revisión editorial un pre-registro
de unas investigaciones, en un número especial para este tipo de
productos. Lo anterior es indicativo del rol que las iniciativas
editoriales cumplen a la hora de incentivar prácticas de transparencia
en el proceso de investigación.

Del mismo modo, ha de destacarse que buena parte de los entrevistados no
conocían el concepto de pre-registro. Los entrevistados familiarizados
con el pre-registro sin practicarlo presentan diversas justificaciones
para ello. Por una parte, se destaca que el pre-registro sólo
corresponde a cierto tipo de investigaciones - particularmente,
cuantitativas con hipótesis -, por lo cual es una práctica que tiende a
carecer de sentido (Psicología, cualitativos; Economía, cuantitativos).
Por otra, se argumenta que el pre-registro de la investigación antes de
su ejecución puede constituir una ``camisa de fuerza'' (Ciencia
política, cualitativos), sobre todo en caso de seguir una lógica
inductiva, dada la dificultad de asegurar una relación estrecha entre lo
diseñado y los hallazgos finales. Se destaca que el diseño sólo existe
en su ejecución (Sociología, cualitativos), por lo cual publicarlo antes
de llevar adelante la investigación sería problemático. Los
investigadores principalmente cualitativos destacan, entonces, las
limitaciones metodológicas a las que se enfrentarían a la hora de
pre-registrar una investigación de tales características.

Junto con las limitaciones metodológicas a la práctica del pre-registro
destacan preocupaciones asociadas a la amenaza que puede implicar para
los derechos de propiedad intelectual el publicar diseños de
investigaciones que todavía no son llevadas a cabo. Si bien hay quienes
destacan la compatibilidad entre el resguardo de la propiedad
intelectual y el pre-registro de los diseños en la medida que esto
último permite asegurar la autoría de una investigación antes de
ejecutarla (Ciencia política, métodos mixtos), otros entrevistados ponen
sobre la mesa su preocupación por perder ideas de investigación
originales. Esto está muchas veces asociado a que gran parte de su
capacidad de publicar en revistas de alto impacto - y, por ende, de
cumplir los lineamientos de la estructura de financiamiento de la
Agencia Nacional de Investigación y Desarrollo (ANID) - se debe a la
originalidad de sus diseños de investigación (Psicología,
cuantitativos). Sumando a ello malas experiencias previas, como la
ejecución de un diseño de investigación original por parte de un centro
de investigación renombrado del Norte global, el pre-registro es una
práctica frente a la cual los investigadores tienden a presentar una
actitud suspicaz.

Pese a destacar el hecho de que el pre-registro de hipótesis permite
asegurar que una investigación no incluye prácticas como el HARKing
(Economía, cuantitativos), el bajo conocimiento y formación en esta
práctica; su asociación a ciertos tipos de investigación; las
limitaciones metodológicas planteadas; y las preocupaciones asociadas a
las presión por publicar que señalan los entrevistados están a la base
de las sospechas y consiguiente generalización de no practicar el
pre-registro.

\begin{enumerate}
\def\labelenumi{\alph{enumi})}
\setcounter{enumi}{2}
\tightlist
\item
  Apertura de datos
\end{enumerate}

Entre los entrevistados predomina una actitud generalmente favorable
respecto a la apertura de datos como iniciativa, sin embargo, sólo dos
entrevistados han disponibilizado sus datos en la práctica ( ). Esta
contradicción se sostiene en creencias de dificultad técnica o
metodológica para la publicación de los datos, la aparente inutilidad de
la apertura de datos, y la tensión que existe entre perder la
exclusividad de uso de los datos y la presión por publicar, suponiendo
que la práctica de apertura de datos constituye un riesgo para la
carrera académica.

En cuanto a las actitudes positivas frente a la apertura de datos, estas
se fundamentan principalmente en la creencia de que esta es útil para el
desarrollo de la ciencia, abriendo la posibilidad de reproducir los
análisis y mejorar la calidad de conocimiento, favoreciendo el impacto
científico de las publicaciones asociadas a los datos, permitiendo
ahorrar recursos humanos y monetarios destinados a la producción de
datos, y reduciendo el impacto que tiene la producción de datos en
comunidades marginadas o grupos sociales estudiados con mayor
frecuencia. Especialmente en el caso de las investigaciones financiadas
con fondos públicos, la mayoría de los investigadores mostraron una
actitud altamente favorable a la apertura de datos, y varios señalaron
que debería ser un requisito más enfático u obligatorio. Algunos
investigadores incluso sostuvieron la idea de que los datos en este tipo
de investigaciones son bienes públicos, y la no disponibilización de los
datos se percibía como un acaparamiento ilegítimo ( quizás reemplazar
por idea del imperativo moral).

Por otro lado, algunos investigadores fueron menos severos con respecto
a la obligatoriedad de la apertura, destacando la inversión personal del
investigador en la producción de los datos y el costo extra que implica
su disponibilización. Además, algunos señalan que los datos son, al
menos por un tiempo, propiedad del investigador, en tanto su producción
requiere trabajo de innovación y creatividad, establecimiento de
relaciones con los sujetos de estudio o instituciones, y su producción
no puede ser entendida solamente como una inversión de recursos en manos
del investigador, sino como producción intelectual. En este sentido,
entra en tensión la apertura de datos y la presión por publicar, no sólo
por el tiempo que el investigador deba dedicar a la apertura, sino
también por la presión por productos científicos con preguntas o diseños
originales y novedosos. Varios investigadores asocian la apertura de los
datos a la pérdida de control sobre su uso, y la posible dificultad de
publicar utilizando datos abiertos. Asimismo, algunos investigadores
cuestionan el sentido de abrir los datos, señalando que es poco común
que otros investigadores usen los datos, sosteniendo que es difícil
utilizar datos públicos de forma novedosa porque ya han sido utilizados,
y que no necesariamente se adecúan a la especificidad de la
investigación en ciencias sociales:

\begin{verbatim}
[...] en la ciencia social [...] la relación con los datos [es] una relación que tiene sentido y que es significativa en la medida que se produce en el marco de estos programas de investigación más amplios. [...] Hay una especie de fantasía, de que si liberamos los datos y están todas las entrevistas y todas las bases de datos, '¡oh sí!', o sea, creo que el trabajo de investigación científica tiene un nivel de complejidad, de especificidad, que haría una alerta [...] sobre la fantasía de que se liberan los datos, las entrevistas, y vamos a comenzar a producir mucho conocimiento muy valioso, muy significativo. (Sociología, cualitativos)
\end{verbatim}

Acompañando esta actitud suspicaz respecto a la utilidad de la apertura
de datos, los entrevistados señalan los costos metodológicos y riesgos
éticos que perciben los investigadores en la práctica. Por un lado, la
totalidad de los investigadores consideran que la mayor parte de los
datos cuantitativos no debieran requerir protocolos especiales de ética
ni condiciones de uso, exceptuando por datos producidos en poblaciones
específicas o en los estudios de economía en que los datos pertenecen a
empresas privadas. Por otro lado, la apertura de datos cualitativos
presenta especial dificultad en el resguardo de la identidad de los
participantes en el trabajo con comunidades específicas, además de
requerir muchos recursos humanos para su anonimización, y el riesgo de
afectación a las comunidades excede su anonimato, siendo también
respecto al contenido de lo que se publica cuando el investigador no
mantiene una relación directa de rapport con la comunidad. Esto a su vez
se relaciona con una dificultad metodológica, ya que los datos
cualitativos al estar anonimizados y descontextualizados, limitan el
análisis de datos secundarios y desincentivan su uso.

Cabe destacar que a pesar de la percepción de diferencias en facilidad y
pertinencia de apertura entre datos cualitativos y cuantitativos, sólo
dos de los investigadores entrevistados, uno cualitativo y uno
cuantitativo, habían practicado alguna vez la publicación abierta de
datos más allá de los requisitos editoriales y de compartir datos
directamente con otros investigadores.

\begin{enumerate}
\def\labelenumi{\alph{enumi})}
\setcounter{enumi}{3}
\tightlist
\item
  Transparencia del análisis:
\end{enumerate}

Debe destacar una valoración general positiva hacia la transparencia de
los análisis, entendida en sentido amplio y diferido según el tipo de
investigación transparentada. En primer lugar, existe una percepción
generalizada respecto de que las revistas de alto impacto están
exigiendo, cada vez más, apartados metodológicos más claros y detallados
para las investigaciones cualitativas; así como la solicitud de los
códigos de análisis para investigación cuantitativas, sobre todo en
economía. En ese sentido, la transparencia del análisis también se
asocia a la rigurosidad de los procedimientos analíticos ejecutados y,
entonces, a la validez de los resultados presentados.

En ese aspecto, se describen prácticas y expectativas distintas en lo
que respecta a la transparencia de los análisis realizados en
investigaciones cualitativas o cuantitativas. En lo que respecta a las
primeras, la transparencia del análisis está sobre todo vinculado con la
claridad del apartado metodológico de la publicación, a lo cual pueden
sumarse anexos metodológicos que incluyan explicaciones más detalladas
respecto de los procedimientos analíticos realizados. No obstante, no se
hace mayor referencia a la publicación de otros insumos, como el libro
de códigos generados, outputs de los softwares utilizados, entre otros.
En ese sentido, no existe para estos investigadores entre transparencia
de los procedimientos de análisis, y expectativas de reproducibilidad ni
replicabilidad de los hallazgos producidos. Más aún, hay quienes
descartan la posibilidad de reproducir los procedimientos de análisis
dada la hiper-contextualización característica de los datos
cualitativos: aun teniendo acceso a los datos y procedimientos de
análisis producidos, la posibilidad de llegar a conclusiones no es tan
clara para los investigadores cualitativos. Asimismo, se considera que
la propia ontología de la sociedad, dinámica y en continua
transformación, sería una gran traba a la hora de replicar diseños de
investigación en realidades sociales distintas, o bien pertenecientes a
otros momentos históricos.

Por su parte, la transparencia de análisis cuantitativos tiende a
asociarse a la disponibilización (pública o privada) de los códigos de
análisis elaborados. Destaca el hecho de que, muchas veces, esta
disponibilización tiende a solicitarse desde las revistas académicas. Si
bien no se tematiza del todo, es posible inferir que la mayor extensión
de la utilización de software para estos procedimientos, en comparación
con lo que sucede en investigación cualitativa, tienda a facilitar la
consolidación de un conjunto de prácticas y actitudes más favorables
hacia la transparencia de los análisis. No obstante, ninguno de los
investigadores entrevistados declaró haber publicado abiertamente su
código de análisis, en tanto estos tienden a solicitarse y compartirse
de manera privada, bien a través de redes sociales académicas como las
expuestas en apartados anteriores; bien a través de contacto personal
entre investigadores. Además, se plantea que los códigos de análisis son
como los ``cepillos de dientes'' (Economía, cuantitativos), en el
sentido de que su elaboración tiende a ser personal y no pensada para
compartirlo abiertamente con otros investigadores. Sumando a ello
complicaciones adicionales respecto de condiciones éticas o
metodológicas que impidan o dificulten el compartir los datos
analizados, se destaca la insuficiencia que conlleva publicar estos
códigos sin un gran cuidado de su orden, claridad, y de los
procedimientos ejecutados, identificando la necesidad de un formato
común en este aspecto. En último lugar, una informante afirma que los
investigadores cuantitativos no han migrado a la utilización de software
de código abierto por el costo que implicaría para ellos, pues el
estándar de disciplinas como la economía está constituido por software
con barreras de pago y no existe el incentivo suficiente para realizar
un cambio masivo en este aspecto.

Un segundo aspecto que está a la base de la valoración transversalmente
positiva frente a la transparencia de los procedimientos de análisis
está en la importancia que esto puede implicar para el avance de la
ciencia y para el aprendizaje metodológico de otros investigadores. Una
investigadora cualitativa rememora cómo académicas de otros países de
América Latina, inspiradas en sus investigaciones, le han escrito para
comentarle que están utilizando sus textos como insumo
teórico-metodológico para elaborar sus propias investigaciones.

\bookmarksetup{startatroot}

\hypertarget{resultados}{%
\chapter{Resultados}\label{resultados}}

\bookmarksetup{startatroot}

\hypertarget{discusiuxf3n}{%
\chapter{Discusión}\label{discusiuxf3n}}

\bookmarksetup{startatroot}

\hypertarget{referencias}{%
\chapter*{Referencias}\label{referencias}}
\addcontentsline{toc}{chapter}{Referencias}

\markboth{Referencias}{Referencias}

\hypertarget{refs}{}
\begin{CSLReferences}{0}{0}
\end{CSLReferences}



\end{document}
